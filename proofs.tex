\documentclass{article}

%%%%%%%%%%%%%%%%%%%%%%%%%%%%%%%%%%%%%%%%%%%%%%%%%
% Imports
%%%%%%%%%%%%%%%%%%%%%%%%%%%%%%%%%%%%%%%%%%%%%%%%%
%\usepackage[english]{babel}
\usepackage[utf8]{inputenc}
\usepackage[backend=biber,sorting=none,hyperref]{biblatex}
\usepackage{mathtools}
\usepackage{dsfont}
\usepackage{tikz}
\usetikzlibrary{topaths,calc,tikzmark}
\usepackage{algorithm2e}


\begin{document}

\section{The new algorithm}
The currently fastest algorithm is pretty good.
It's well thought out and quite efficient.
But the algorithm searches a lot of redundant space.
Some colorings have to have the same validity as others and search them as many times as there are duplicates ($r!$) is a waste.
The key insight is that we can end some searches earlier if we know we eventually reach a coloring with equivalent validity.


There are some different kinds of colorings that I've decided to group in a very specific way.
We always start at the nodes that match the \textit{top-regex}, I call these \textit{top-coloring}.

Consider that we use the colors $a, b, c, d \dots$. We order them alphabetically which gives us an ordering of the coloring.
The \textit{top-regex} requires all colors and exactly one of each color in ``rising`` order with the first colors spread between them.
All colors also have to be present, so we only allow surjective colorings.
\\
\textit{top-regex}: $aa*ba*ca*\dots$ \hspace{1em} \textit{e.g.}: $abc$, $abaca$, $aabaacaa$
\\
There are also equivalence classes, where colorings which have to be valid at the same time are mapped to.
Consider the colorings $abc$ and $cba$. No matter the edge configuration of the hyper graph, these two colorings have to share validity.
This is because we can rename each color to get the other color -- colorings only care about the internal equality of colors.
We can then send all of these to their representative by renaming the colors. We start from the left and rename all colors, calling
the first color we see $a$, the second $b$ and so on. We can also describe a regex that matches this.
\\
\textit{eq-regex}: $a[a]*b[ab]*c[abc]*\dots$ \hspace{1em} \textit{e.g.}: $abc$, $ababacaa$, $aabaccab$
\\
Note specially that if \textit{top-regex} matches, then \textit{eq-regex} also matches. All top-nodes are eq-nodes, \textit{top-reqeg} implies \texxtbf{eq-regex}.
\textit{eq-regex} matching is a requirement for \textit{top-regex} to match.

\begin{verbatim}
DetNRC(H):
  foreach top_coloring(c, F) do
    if DetLocalSearch(H, c, F, n - r) then
      return YES
  return NO

isRepresentativeColoring(c):
  mapping := empty
  q := 0
  for i in stableNodeOrder(c):
      if c(i) is not mapped in mapping then
        mapping(c(i)) := q
        q := q + 1
  return true if mapping describes the same coloring as c

DetLocalSearch(H, c, F, g):
  -- This is the secret sauce
  if not isRepresentativeColoring(c) then return NO

  if g = 0 and induces_raindbow_edge(c, H) then return NO

  if induces_raindbow_edge(c, H).issubset(F) then return NO

  if is_no_rainbow_coloring(c, H) then return YES

  if |e intersect F| != r - 1 forall e in E then return YES

  e' = pick_any(e where |e intersect F| == r - 1)
  v = pick_any(v in (e' - F))
  foreach j in [r] - {c(v)} do
    if DetLocalSearch(H, c but c(v) = j, F union {v}) then return YES

  return NO
\end{verbatim}

\subsection{Some intuition} 

\pagebreak

\section{Proofs}

\subsection{All equivalence classes are within the edit distance of $n - r$ from their corresponding \textit{top-node}}
Pick a random surjective coloring $X$. We can now send it to it's representative by renaming the colors, calling the new color $X'$.
We know $X'$ matches the \textbf{eq-regex}, since it is surjective and a representation.

If we now change all but the first appearance of each color to the $a$-color, we have something that matches a \textit{top-node}.
This means the edit distance from any surjective coloring to its corresponding \textit{top-node} is at most $n - r$.

\subsubsection{Some clarification and an example}
Since the first appearance of a color in the coloring tuple has to be the same -- we can at most replace all other colors.
Consider $aabbcc$, the \textit{top-node} corresponds to $aabaca$.

\subsection{All equivalence classes are within the edit distance of $n - r$}

\end{document}
