\documentclass{article}
\usepackage[english]{babel}
\usepackage[utf8]{inputenc}
\usepackage[backend=biber,style=apa]{biblatex}
\usepackage{mathtools}
\usepackage{dsfont}

\addbibresource{edvth289_extended_thesis_plan_seminar_2.bib}

\title{A faster algorithm for the no-rainbow problem}

\author{Edvard Thörnros}

\begin{document}
\maketitle

\begin{abstract}

No Rainbow is a constraint satisfaction problem.
The No Rainbow problem is finding a surjective node coloring of an r-regular multigraph.
This thesis extends \cite{sourceNoRainbow}.
A faster algorithm using equivalence classes is suggested, evaluated and proved to be correct.

Other ideas for potential algorithms are also suggested.

\end{abstract}

\section{Introduction}
The faster something can be done, the faster technology can iterate.
A part of this is finding faster algorithms for known hard problems.
One such hard problem is the no rainbow problem, which is NP-complete\cite{sourceNoRainbow}.
The no rainbow problems asks if there is a surjective node coloring of an $r$-regular multigraph.

Rephrased, the no rainbow problem asks if we can assign each node a color.
In a graph with $r$ nodes per edge -- each edge connects $r$ different nodes.
All $r$ colors should be represented in the entire graph.
No edge should connect all $r$ colors.
A node coloring which satisfies these constraints is a solution to the no rainbow problem.

An edge which connects all $r$ colors is called a rainbow edge -- hence the name of the problem.
The no rainbow problem is a variant of the constraint satisfaction problem.

The no rainbow problem is related to the phylogenetic decisiveness problem. \cite{sourcePhylogeneticDecisiveness}

% TODO: Read https://ieeexplore.ieee.org/document/9616390
% TODO: https://www.researchgate.net/publication/350673538_Exact_Algorithms_for_No-Rainbow_Coloring_and_Phylogenetic_Decisiveness

\subsection{Research questions}
\begin{enumerate}
  \item Is there a faster deterministic algoritm that solves the no rainbow problem than the one suggested by \cite{sourceNoRainbow}?
  \item Is there a faster randomized algoritm that solves the no rainbow problem than the one suggested by \cite{sourceNoRainbow}?
  \item How can a solution to the no rainbow problem be found, understood, modeled and solved?
  \item What insights are found by studying the no rainbow problem?
  \item Is it possible that there are even faster algorithms that solve the no rainbow problem?
\end{enumerate}

\section{Background}
The no rainbow problem is quite complex and can be understood in a lot of different ways.
To understand the problem as broadly as possible a few different areas of discrete mathematics are needed.

\subsection{Equivalence classes and equivalence relations}
Equivalence classes are used to group things that are equal in some sense.
This requires we have a set of objects and an binary operator we can use to check if they are equal.
In this general case we use $\sim$ as the operator.

$$
  a \sim b \implies \textit{$a$ is related to $b$ under the equivalence relation ($\sim$)}
$$

A common way of defining equivalence relations is to map the objects to their image and then use equality.
More generally we can define it using:
$$
  a \sim b \iff f(a) = f(b)
$$
Where $f(a)$ is the image of $a$.

For example:
$$
  a \sim b \iff (a \bmod 3) = (b \bmod 3) : \quad a, b \in \mathds{Z}
$$

Here we send each object to their image -- which is their remained after division by 3 -- and then compare them.
Here we get 3 equivalence classes $\{[0], [1], [2]\}$. And all integers are mapped into one of these 3 class.

The representative of a class is the element which has itself as image. In other words, $f(a) = a$ iff $a$ is a representative.
\cite{sourceArmen} \cite{sourceAATA}

\subsection{Multi Graphs}

\subsection{Graph colorings}

\subsection{Constraint statisfaction and related algoritms}

\subsection{NP completeness}

\printbibliography

\end{document}
