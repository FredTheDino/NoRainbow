\chapter{Introduction}
The faster something can be done, the faster technology can iterate.
A part of this is finding faster algorithms for known hard problems.
One such hard problem is the no rainbow problem, which is NP-complete.
The no rainbow problems asks if there is a surjective node coloring of an $r$-regular hypergraph.
\cite{sourceNoRainbow}

Rephrased, the no rainbow problem asks if we can assign each node a color.
In a graph with $r$ nodes per edge -- each edge connects $r$ different nodes.
All $r$ colors should be represented in the entire graph.
No edge should connect all $r$ colors.
A node coloring which satisfies these constraints is a solution to the no rainbow problem.

An edge which connects all $r$ colors is called a rainbow edge -- hence the name of the problem.
The no rainbow problem is a variant of the constraint satisfaction problem.

The no rainbow problem is related to the phylogenetic decisiveness problem. \cite{sourcePhylogeneticDecisiveness}
% TODO: Read https://ieeexplore.ieee.org/document/9616390
% TODO: https://www.researchgate.net/publication/350673538_Exact_Algorithms_for_No-Rainbow_Coloring_and_Phylogenetic_Decisiveness

\section{Research questions}
\begin{enumerate}
  \item Is there a faster deterministic algorithm that solves the no rainbow problem than the one suggested by Ghazaleh Parvini and David Fernandez-Baca \cite{sourceNoRainbow}?
  \item Is there a faster randomized algorithm that solves the no rainbow problem than the one suggested by Ghazaleh Parvini and David Fernandez-Baca \cite{sourceNoRainbow}?
  \item How can a solution to the no rainbow problem be found, understood, modeled and presented?
  \item What insights into algorithms are found by studying the no rainbow problem?
  \item How fast is the fastest possible algorithm for solving the no rainbow problem? 
\end{enumerate}
